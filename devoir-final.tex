% Options for packages loaded elsewhere
\PassOptionsToPackage{unicode}{hyperref}
\PassOptionsToPackage{hyphens}{url}
%
\documentclass[
]{article}
\usepackage{amsmath,amssymb}
\usepackage{lmodern}
\usepackage{iftex}
\ifPDFTeX
  \usepackage[T1]{fontenc}
  \usepackage[utf8]{inputenc}
  \usepackage{textcomp} % provide euro and other symbols
\else % if luatex or xetex
  \usepackage{unicode-math}
  \defaultfontfeatures{Scale=MatchLowercase}
  \defaultfontfeatures[\rmfamily]{Ligatures=TeX,Scale=1}
\fi
% Use upquote if available, for straight quotes in verbatim environments
\IfFileExists{upquote.sty}{\usepackage{upquote}}{}
\IfFileExists{microtype.sty}{% use microtype if available
  \usepackage[]{microtype}
  \UseMicrotypeSet[protrusion]{basicmath} % disable protrusion for tt fonts
}{}
\makeatletter
\@ifundefined{KOMAClassName}{% if non-KOMA class
  \IfFileExists{parskip.sty}{%
    \usepackage{parskip}
  }{% else
    \setlength{\parindent}{0pt}
    \setlength{\parskip}{6pt plus 2pt minus 1pt}}
}{% if KOMA class
  \KOMAoptions{parskip=half}}
\makeatother
\usepackage{xcolor}
\usepackage[margin=1in]{geometry}
\usepackage{graphicx}
\makeatletter
\def\maxwidth{\ifdim\Gin@nat@width>\linewidth\linewidth\else\Gin@nat@width\fi}
\def\maxheight{\ifdim\Gin@nat@height>\textheight\textheight\else\Gin@nat@height\fi}
\makeatother
% Scale images if necessary, so that they will not overflow the page
% margins by default, and it is still possible to overwrite the defaults
% using explicit options in \includegraphics[width, height, ...]{}
\setkeys{Gin}{width=\maxwidth,height=\maxheight,keepaspectratio}
% Set default figure placement to htbp
\makeatletter
\def\fps@figure{htbp}
\makeatother
\setlength{\emergencystretch}{3em} % prevent overfull lines
\providecommand{\tightlist}{%
  \setlength{\itemsep}{0pt}\setlength{\parskip}{0pt}}
\setcounter{secnumdepth}{-\maxdimen} % remove section numbering
\ifLuaTeX
  \usepackage{selnolig}  % disable illegal ligatures
\fi
\IfFileExists{bookmark.sty}{\usepackage{bookmark}}{\usepackage{hyperref}}
\IfFileExists{xurl.sty}{\usepackage{xurl}}{} % add URL line breaks if available
\urlstyle{same} % disable monospaced font for URLs
\hypersetup{
  pdftitle={Devir final Mr},
  pdfauthor={Joël Mugambi Nicolas, Ralf Oly Bertilus,Janvier Gerald Euclide,Leger Michelson},
  hidelinks,
  pdfcreator={LaTeX via pandoc}}

\title{Devir final Mr}
\author{Joël Mugambi Nicolas, Ralf Oly Bertilus,Janvier Gerald
Euclide,Leger Michelson}
\date{2023-06-17}

\begin{document}
\maketitle

\#On voulait expliquer le prduit national d'Haiti pour la periode allant
de 1987 a 2021, on a choisi trois variables qui sont l'investissement
direct etranger, le commerce net qu'on designe comme l'exportation nette
et le taux d'inflation. on attend en relation positive entre les
exportations nettes et le taux de croissance du PIB, tout comme entre
les investissements directs etrangers. mais entre le taux d'inflation et
le taux de croissance du PIB il existe une relation negative. on a
decide de choisir ces trois variables parce qu'on pense qu'elles ont
beaucoup a faire dans la mauvaise performance de l'econmie haitienne.
\#les investissements directs etrangers devraient ameliorer l'economie
en mettant a la disposition des agents economiques plus de fonds. Les
exportations nettes devraient apporter de devises dans l'economie mais
on attend a n'importe quoi puisqu'on sait que l'economie importe plus
q'elle exporte. Pour le taux d'inflation, il devrait influence le PIB
negativement puisqu'il fait diminuer le pouvoir d'achat des agents des
lors il decourage les demandeurs. '\,'

'\,'\,'\{r main, include=FALSE\} library(wbstats) \#Importation
taux\_croiss \textless- wb\_data(indicator = ``NY.GDP.MKTP.KD.ZG'' ,
country = ``HTI'' , start\_date = 1987 , end\_date = 2021) IDE
\textless- wb\_data(indicator = ``BX.KLT.DINV.WD.GD.ZS'' , country =
``HTI'' , start\_date = 1987 , end\_date = 2021) inflation \textless-
wb\_data(indicator = ``FP.CPI.TOTL.ZG'' , country = ``HTI'' ,
start\_date = 1987 , end\_date = 2021) commerce\_net \textless-
wb\_data(indicator = ``BN.GSR.GNFS.CD'' , country = ``HTI'' ,
start\_date = 1987 , end\_date = 2021)

\#Cleaning up taux\_croissance \textless- taux\_croiss{[} ,c(3,4,5){]}
IDE \textless- IDE{[} ,c(5){]} commerce\_net \textless- commerce\_net{[}
,c(5){]} inf \textless- inflation{[} ,c(5){]}

\#Forme matricielle des donnees mat \textless-
cbind(taux\_croissance,IDE,commerce\_net,inf)

\#Renommons les colonnes colnames(mat) \textless-
c(``Pays'',``Annee'',``Taux\_de\_croissance'',``Investissement\_direct\_etranger'',``Exportations\_nettes'',``Taux\_dinflation'')

\#Realisons la regression lineaire reg\_dj \textless-
lm(Taux\_de\_croissance\textasciitilde Investissement\_direct\_etranger+Exportations\_nettes+Taux\_dinflation,mat)
reg\_gt \textless- summary(reg\_dj) reg\_coefficient
\textless-coefficients(reg\_gt) residu\_de\_regression
\textless-residuals(reg\_gt)

\#Trouver f statistics ftics
\textless-reg\_gt\(fstatistic rtics <-reg_gt\)r.squared stats\_tab
\textless- cbind(ftics,rtics,reg\_coefficient) '\,'\,'

summary(cars) ``` \#Installation du package car library(car)

\#Realisons un nuage de points pour chacune des variables independantes
scatterplot(Taux\_de\_croissance\textasciitilde Investissement\_direct\_etranger,mat,regLine=list(method=lm,lty=1,lwd=2,col=``yellow''))
scatterplot(Taux\_de\_croissance\textasciitilde Exportations\_nettes,mat,regLine=list(method=lm,lty=1,lwd=2,col=``orange''))
scatterplot(Taux\_de\_croissance\textasciitilde Taux\_dinflation,mat,regLine=list(method=lm,lty=1,lwd=2,col=``green''))

\#Realiser un graphique en nuage de points mettant en lien les valeurs
residuelles et les valeurs estimeees plot(reg\_dj,which=1)

\#CE graphe de nuge de points des valeurs residuelles(difference entre
valeurs observees et valeur estimes)est a tendance positve et non
lineaire.Ce graphe presente une homocedasticite des points, c est a dire
a dispersion egale.

\includegraphics{devoir-final_files/figure-latex/pressure-1.pdf}

Note that the \texttt{echo\ =\ FALSE} parameter was added to the code
chunk to prevent printing of the R code that generated the plot.

\end{document}
